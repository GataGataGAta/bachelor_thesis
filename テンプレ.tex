% Last updated at 2026/01/12 13:43:01
\documentclass[12pt,a4j]{ltjsreport}
\usepackage{ltklthesis}

\begin{document}

\type{B} %卒論はB,修論はM
\author{北村 泰彦}
\id{12345678} %学籍番号
\gyear{2026} %修了・卒業年(修了・卒業式が2025年3月なら2025)
\title{論文の書き方\\(タイトルに改行をいれる場合は\\単語の途中にいれない)}
\abstract{
  従来の研究,目的,手法,結果を簡潔に書く.
  500字以内に収めること.
  図表は入れない.参考文献は参照しない.
}

\maketitle

\chapter{はじめに}

研究分野(研究背景,関連研究),研究目的,研究方法,評価の方法,論文構成を簡潔に書く.研究分野には,研究背景として社会的な意義(どう世の中の役に立つ研究なのか?),関連研究の概要について書く.論文タイトルに含まれるキーワードに関する説明は必要である.節に分けない.図表は入れない.過去形ではなく現在形で書く.標準的に1~2ページ程度.

論文執筆を始める前に,「論文の書き方」の資料に目を通すこと.

文章の添削をChatGPTなどの生成AIに依頼することも有用である.

\chapter{研究分野を表す適切なタイトル}
研究分野に関して,他人がやったこと書く.

\section{研究分野を表すタイトル}

研究分野における問題の定義,専門用語の説明.

\section{要素技術}

要素技術の説明.3章で書いてもよい.

\section{関連研究}

従来の研究と課題.

\subsection{関連研究1}

\subsection{関連研究2}

\section{本研究の目的}

関連研究の課題への対応する研究目的に関して詳しく書く.


\chapter{研究目的を表す適切なタイトル}

手法やアルゴリズムを開発するのであれば,その処理の手順を示す.各処理の詳細を書く.具体例を示す.

システムを開発する場合は,設計方針(目的達成に必要な機能の一覧),全体構成と各モジュールの役割や関係,各モジュールの詳細,システムの利用例,について書く.モジュールの説明においては,入力と出力を明確にすること.

自分がやったことを書く.

\section{設計方針}

\section{全体構成}

\section{モジュール1の詳細}

\section{モジュール2の詳細}

\section{システムの利用例}


\chapter{評価}

評価の目的,方法,結果,考察(研究の目的がどこまで達成できたのかについて),今後の課題について書く.
実験結果はグラフや表にまとめる.
実験結果に関しては必要に応じて,統計的検定を行う.
考察に関してはその根拠となる実験結果を示すこと.

\section{評価の目的}
\section{評価の方法}
\section{評価の結果} 
\section{考察}
\section{今後の課題}

\chapter{まとめ}

研究の目的と結果を簡潔に書く.
今後の課題に関して簡潔に書く.

\appendix
\chapter{LaTexの書き方}

\section{図の挿入}
図\ref{fig:sample_eps},図\ref{fig:sample_pdf},図\ref{fig:sample_svg}のように,図を挿入するときは必ずfigure環境を用い,label, captionをつけること.
図は必ず本文中で引用し,説明を加えること.図はeps, pdf, またはsvg形式にすること.
パワポで図を書く場合は,InkScapeに図をコピペし,eps, pdf, またはsvg形式で保存すること.

\begin{figure}[htbp]
  \begin{center}
    \includegraphics[width=6cm]{image/sample.eps}
    \caption{サンプル(eps形式)}
    \label{fig:sample_eps}
  \end{center}
\end{figure}

\begin{figure}[htbp]
  \begin{center}
    \includegraphics[width=6cm]{image/sample.pdf}
    \caption{サンプル(pdf形式)}
    \label{fig:sample_pdf}
  \end{center}
\end{figure}

\begin{figure}[htbp]
  \begin{center}
    \includesvg[width=6cm]{image/sample.svg}
    \caption{サンプル(svg形式)}
    \label{fig:sample_svg}
  \end{center}
\end{figure}

\section{表の挿入}
表\ref{tbl:q1_macro}のように,表を挿入するときは必ずtable環境を用い,label, captionをつけること.
表は必ず本文中で引用し,説明を加えること.

\begin{table}[htbp]
  \caption{クラスタ数と選択肢の割合}
  \label{tbl:q1_macro}
  \begin{center}
    \begin{tabular}{lrrrrr}
      \hline
      \multirow{2}{*}{クラスタ数} & \multicolumn{4}{c}{結果 } & \multirow{2}{*}{計}                   \\
                             & 当然                      & おもしろい              & 無意味 & 無回答 &     \\ \hline
      1(10例)                 & 189                     & 233                & 46  & 2   & 470 \\
      2(8例)                  & 110                     & 219                & 43  & 4   & 376 \\
      3(2例)                  & 22                      & 60                 & 12  & 0   & 94  \\ \hline
      計                      & 321                     & 512                & 101 & 6   & 940 \\
    \end{tabular}
  \end{center}
\end{table}

ChatGPTを活用すると図\ref{fig:table_png}のような画像形式の表を表\ref{tbl:table_tex}のようなLaTeX形式の表に変換してくれる.ただし,誤りも含まれるので適宜,修正が必要である.

\begin{figure}[h]
  \begin{center}
    \includegraphics[width=10cm]{image/table.png}
    \caption{画像の表(png形式)}
    \label{fig:table_png}
  \end{center}
\end{figure}

\begin{table}[h]
  \centering
  \renewcommand{\arraystretch}{1.5}
  \begin{tabular}{|c|c|c|}
    \hline
    \multicolumn{2}{|c|}{腕を使った技} & 脚を使った技             \\ \hline
    拳を使った技                       & 拳以外を使った技 &         \\ \hline
    中段突き                         & 肘打ち      & 前蹴り上げ   \\ \hline
    上段突き                         & 肘打ち上げ    & 内回し蹴り   \\ \hline
    裏拳左右打ち                       & 肘打ち下し    & 外回し蹴り   \\ \hline
    裏拳正面打ち                       & 手刀鎌面打ち   & 中段前蹴り   \\ \hline
    裏拳回し打ち                       & 手刀鎌骨打ち込み & 金的蹴り    \\ \hline
    正拳顔打ち                        & 手刀内打ち    & 足刀横蹴り   \\ \hline
    上段受け                         & 手刀腱横打ち   & 足刀横蹴り上げ \\ \hline
    外受け                          &          & 膝蹴り     \\ \hline
    下段払い                         &          & 後ろ蹴り    \\ \hline
    円形逆突き                        &          & 上段回し蹴り  \\ \hline
    内突き                          &          &         \\ \hline
    内突き下段払い                      &          &         \\ \hline
  \end{tabular}
  \caption{腕と脚を使った技の分類}
  \label{tbl:table_tex}
\end{table}

\section{コードの挿入}
Code~\ref{program1}のように,ソースコードを挿入する場合はlstlisting環境を使い,label, captionをつけること.

\begin{minipage}{14cm}
  \begin{lstlisting}[caption=Pythonのソースコード,label=program1]
import time

def main():
time.sleep(5)  # スリープします
print("ok")

# testtest
if __name__ == "__main__":
main()
\end{lstlisting}
\end{minipage}

Code~\ref{program2}のように,ソースコードをファイルから挿入する場合はlstinputlisting環境を使う.

\begin{minipage}{14cm}
  \lstinputlisting[caption = Pythonのソースコード2 ,label = program2]{code/sample.py}
\end{minipage}

\section{参考文献}
参考文献はbibtexを使って最後に書く.
参考文献は以下のように本文中で引用する.
書籍は\cite{latex},論文は\cite{高橋12},国際会議論文は\cite{Nakanishi16}, 研究会報告は\cite{岡本22}, 学会全国大会発表は\cite{芦田21}, 卒業論文は\cite{増田19}, Webページは\cite{北村研究室}のように書く.

\acknowledgement %謝辞

本研究を進めるにあたり,終始ご指導下さったXX教授に深く感謝いたします.

また,音声認識に関して協力していただいた知能機械工学課程YY教授に厚くお礼申し上げます.
ほかに協力してもらった人も書いておく.

北村研究室の同輩諸氏には,
日頃から多くのご助言やご支援をいただき,大変感謝しています.

最後に,4年間大学に通わせてくれた両親に,心から感謝しています.

\reference{sample} %bibファイル名を指定

\end{document}
